\documentclass[a4paper,12pt]{article}
\usepackage[utf8]{inputenc}
\usepackage[italian]{babel}
\usepackage{geometry}
\usepackage{graphicx}
\usepackage{hyperref}
\usepackage{fancyhdr}
\usepackage{listings}
\usepackage{xcolor}
\usepackage{float}
\usepackage{titlesec}

% Configurazione pagina
\geometry{top=2.5cm, bottom=2.5cm, left=2.5cm, right=2.5cm}

% Configurazione Header e Footer
\pagestyle{fancy}
\fancyhf{}
\rhead{Josh V. E. Abanico - CS0525IT}
\lhead{\textbf{Progetto S6/L5 - Authentication Cracking}}
\cfoot{\thepage}

% Configurazione colori per il codice
\definecolor{codegray}{rgb}{0.5,0.5,0.5}
\definecolor{codepurple}{rgb}{0.58,0,0.82}
\definecolor{backcolour}{rgb}{0.95,0.95,0.92}

\lstdefinestyle{mystyle}{
    backgroundcolor=\color{backcolour},   
    commentstyle=\color{codegray},
    keywordstyle=\color{magenta},
    numberstyle=\tiny\color{codegray},
    stringstyle=\color{codepurple},
    basicstyle=\ttfamily\footnotesize,
    breakatwhitespace=false,         
    breaklines=true,                 
    captionpos=b,                    
    keepspaces=true,                 
    numbers=left,                    
    numbersep=5pt,                  
    showspaces=false,                
    showstringspaces=false,
    showtabs=false,                  
    tabsize=2
}
\lstset{style=mystyle}

\title{
    \vspace{2cm}
    \textbf{\huge Report Tecnico: Authentication Cracking} \\
    \vspace{0.5cm}
    \Large Analisi di vulnerabilità mediante attacchi a dizionario su protocolli di rete (SSH/FTP) \\
    \vspace{1cm}
    \textbf{Corso:} Cyber Security \& Ethical Hacking \\
    \textbf{Modulo:} U2 S6 L5
}
\author{\textbf{Josh Van Edward Abanico} \\ CS0525IT}
\date{16 Gennaio 2026}

\begin{document}

\maketitle
\thispagestyle{empty}
\newpage

\tableofcontents
\newpage

% -------------------------------------------------------------------
% SEZIONE 1: INTRODUZIONE
% -------------------------------------------------------------------
\section{Introduzione e Obiettivi}

Il presente report documenta l'attività di laboratorio svolta nell'ambito del progetto S6/L5. L'obiettivo principale è dimostrare la vulnerabilità dei servizi di rete configurati con credenziali deboli attraverso l'utilizzo di strumenti di \textit{password cracking} automatizzati.

Nello specifico, l'analisi si concentra su:
\begin{itemize}
    \item \textbf{Fase 1:} Attacco al servizio SSH (Secure Shell) tramite attacco a dizionario.
    \item \textbf{Fase 2:} Configurazione e attacco al servizio FTP (File Transfer Protocol).
\end{itemize}

Lo strumento principale utilizzato per l'audit è \textbf{Hydra}, un software di login cracker parallelizzato che supporta numerosi protocolli.

% -------------------------------------------------------------------
% SEZIONE 2: METODOLOGIA
% -------------------------------------------------------------------
\section{Metodologia e Setup}

L'ambiente di test è stato configurato utilizzando una macchina attaccante (Kali Linux) e un target configurato localmente all'interno della stessa rete virtuale.

\subsection{Specifiche del Target}
\begin{itemize}
    \item \textbf{IP Target:} 192.168.50.151
    \item \textbf{Servizi Attivi:} SSH (Porta 22), FTP (Porta 21 - in configurazione).
    \item \textbf{Liste utilizzate:} Sono state impiegate le wordlist della collezione \textit{Seclists} (xato-usernames e xato-passwords) per simulare uno scenario realistico di brute-force non mirato.
\end{itemize}
\subsection{Preparazione e Ottimizzazione delle Wordlist}
Poiché le wordlist originali della raccolta \textit{Seclists} (Xato-net-10-million) contengono milioni di voci, l'utilizzo diretto avrebbe richiesto tempi di esecuzione eccessivi per questa simulazione.

È stato quindi eseguito un filtraggio preventivo per creare dizionari ridotti e mirati. Utilizzando i comandi \texttt{grep} e \texttt{head}, sono state estratte le prime 15 occorrenze contenenti la stringa "test" (coerente con l'account target \texttt{test\_user}).

\begin{figure}[H]
    \centering
    \includegraphics[width=\linewidth]{screen_grep.png}
    \caption{Screen del terminale: comandi utilizzati per filtrare e ridirezionare l'output nelle nuove wordlist.}
    \label{fig:wordlist_prep}
\end{figure}

Questa operazione ha generato due file di testo leggeri \textit{xato-passwords.txt} e \textit{xato-usernames.txt} utilizzati successivamente da Hydra.

\newpage
% -------------------------------------------------------------------
% SEZIONE 3: ANALISI TECNICA - SSH
% -------------------------------------------------------------------
\section{Fase 1: Attacco al servizio SSH}

Nella prima fase, è stato verificato il livello di sicurezza del servizio SSH. Dopo aver confermato che il servizio fosse attivo e raggiungibile, è stato lanciato un attacco utilizzando Hydra.

\subsection{Esecuzione dell'Attacco}
Il comando lanciato da terminale è il seguente:

\begin{lstlisting}[language=bash]
hydra -L xato-usernames.txt -P xato-passwords.txt 192.168.50.151 -t 2 ssh -V
\end{lstlisting}

\begin{itemize}
    \item \textbf{-L / -P:} Indica l'uso di liste di username e password (input massivo).
    \item \textbf{-t 2:} Limita il numero di task (thread) a 2 per evitare di sovraccaricare il servizio o causare un blocco immediato.
    \item \textbf{-V:} Modalità "Verbose" per visualizzare i tentativi in tempo reale.
\end{itemize}

\subsection{Evidenze dell'Attacco}

Di seguito viene mostrato l'avvio della procedura di cracking. Hydra inizia a combinare gli username e le password presenti nelle liste fornite.

\begin{figure}[H]
    \centering
    \includegraphics[width=\linewidth]{screen_hydra0.png}
    \caption{Avvio dell'attacco Hydra contro l'IP 192.168.50.151. Si notano i tentativi di login falliti iniziali.}
    \label{fig:hydra0}
\end{figure}

Durante l'esecuzione, lo strumento itera attraverso le combinazioni. Come evidenziato nello screenshot successivo, Hydra identifica una corrispondenza valida per l'utente \texttt{test\_user}.

\begin{figure}[H]
    \centering
    \includegraphics[width=\linewidth]{screen_hydra1.png}
    \caption{Rilevamento delle credenziali valide. La riga evidenziata in verde mostra: login: \textbf{test\_user}, password: \textbf{testpass}.}
    \label{fig:hydra1}
\end{figure}

Al termine della scansione delle liste o al ritrovamento delle credenziali (a seconda della configurazione), Hydra termina l'esecuzione fornendo un riepilogo del successo.

\begin{figure}[H]
    \centering
    \includegraphics[width=\linewidth]{screen_hydra2.png}
    \caption{Conclusione dell'attacco: "1 valid password found". L'attacco ha avuto successo.}
    \label{fig:hydra2}
\end{figure}

% -------------------------------------------------------------------
% SEZIONE 4: ANALISI TECNICA - FTP
% -------------------------------------------------------------------
\section{Fase 2: Configurazione e Attacco FTP}

Nella seconda fase dell'esercitazione, l'attenzione si è spostata sul protocollo di trasferimento file (FTP). A differenza di SSH, FTP trasmette le credenziali in chiaro (se non configurato in modalità sicura), ma è comunque soggetto ad attacchi di brute-force sull'autenticazione.

\subsection{Configurazione del Servizio Target}
Sulla macchina target è stato installato e attivato il demone \texttt{vsftpd} (Very Secure FTP Daemon), standard de facto per i sistemi Linux.

\begin{lstlisting}[language=bash]
sudo apt install vsftpd
sudo service vsftpd start
\end{lstlisting}

Una volta confermato che il servizio fosse in ascolto sulla porta standard TCP/21, è stato preparato l'attacco.

\subsection{Esecuzione dell'Attacco con Hydra}
Utilizzando le stesse wordlist ottimizzate nella fase precedente (\texttt{xato-usernames.txt} e \texttt{xato-passwords.txt}), è stato lanciato Hydra specificando il protocollo \texttt{ftp}.

\begin{lstlisting}[language=bash]
hydra -L xato-usernames.txt -P xato-passwords.txt 192.168.50.151 ftp -V
\end{lstlisting}

\subsection{Evidenze dell'Attacco FTP}

\begin{figure}[H]
     \centering
    \includegraphics[width=\linewidth]{screen_ftp0.png}
     \caption{Esecuzione dell'attacco Hydra contro il servizio FTP.}
 \end{figure}
\subsection{Risultato}
L'attacco ha avuto successo in tempi brevi grazie all'utilizzo delle liste filtrate. Hydra ha completato l'handshake e validato la combinazione corretta di username e password, permettendo l'accesso ai file del server.
 \begin{figure}[H]
     \centering
      \includegraphics[width=\linewidth]{screen_ftp1.png}
     \caption{Credenziali FTP individuate con successo.}
 \end{figure}

% -------------------------------------------------------------------
% SEZIONE 5: RACCOMANDAZIONI
% -------------------------------------------------------------------
\newpage
\section{Raccomandazioni e Mitigazione}

L'esercitazione dimostra quanto sia banale compromettere un sistema che utilizza password deboli o presenti in dizionari comuni (come "testpass"). Per mitigare questi rischi in un ambiente di produzione, si raccomanda di:

\begin{enumerate}
    \item \textbf{Enforce Strong Passwords:} Imporre policy che richiedano password complesse, lunghe e non presenti in wordlist pubbliche.
    \item \textbf{Rate Limiting e Fail2Ban:} Implementare strumenti come \textit{Fail2Ban} che bloccano temporaneamente gli IP dopo un numero definito di tentativi di login falliti, rendendo inefficaci gli attacchi brute-force.
    \item \textbf{Autenticazione a Chiave Pubblica (SSH):} Disabilitare l'autenticazione via password per SSH e utilizzare esclusivamente chiavi RSA/Ed25519.
    \item \textbf{Cambio Porte Standard:} Spostare i servizi dalle porte standard (22, 21) a porte non standard per evitare scansioni automatiche superficiali (security by obscurity, utile solo come misura aggiuntiva).
\end{enumerate}

\section{Conclusioni}
Il report ha confermato la criticità delle configurazioni di default e l'importanza di password robuste. L'attacco ha permesso di ottenere accesso completo al sistema target in pochi minuti.

\end{document}
